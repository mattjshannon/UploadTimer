%% Generated by Sphinx.
\def\sphinxdocclass{report}
\documentclass[letterpaper,10pt,english]{sphinxmanual}
\ifdefined\pdfpxdimen
   \let\sphinxpxdimen\pdfpxdimen\else\newdimen\sphinxpxdimen
\fi \sphinxpxdimen=.75bp\relax

\usepackage[utf8]{inputenc}
\ifdefined\DeclareUnicodeCharacter
 \ifdefined\DeclareUnicodeCharacterAsOptional
  \DeclareUnicodeCharacter{"00A0}{\nobreakspace}
  \DeclareUnicodeCharacter{"2500}{\sphinxunichar{2500}}
  \DeclareUnicodeCharacter{"2502}{\sphinxunichar{2502}}
  \DeclareUnicodeCharacter{"2514}{\sphinxunichar{2514}}
  \DeclareUnicodeCharacter{"251C}{\sphinxunichar{251C}}
  \DeclareUnicodeCharacter{"2572}{\textbackslash}
 \else
  \DeclareUnicodeCharacter{00A0}{\nobreakspace}
  \DeclareUnicodeCharacter{2500}{\sphinxunichar{2500}}
  \DeclareUnicodeCharacter{2502}{\sphinxunichar{2502}}
  \DeclareUnicodeCharacter{2514}{\sphinxunichar{2514}}
  \DeclareUnicodeCharacter{251C}{\sphinxunichar{251C}}
  \DeclareUnicodeCharacter{2572}{\textbackslash}
 \fi
\fi
\usepackage{cmap}
\usepackage[T1]{fontenc}
\usepackage{amsmath,amssymb,amstext}
\usepackage{babel}
\usepackage{times}
\usepackage[Bjarne]{fncychap}
\usepackage[dontkeepoldnames]{sphinx}

\usepackage{geometry}

% Include hyperref last.
\usepackage{hyperref}
% Fix anchor placement for figures with captions.
\usepackage{hypcap}% it must be loaded after hyperref.
% Set up styles of URL: it should be placed after hyperref.
\urlstyle{same}
\addto\captionsenglish{\renewcommand{\contentsname}{Contents:}}

\addto\captionsenglish{\renewcommand{\figurename}{Fig.}}
\addto\captionsenglish{\renewcommand{\tablename}{Table}}
\addto\captionsenglish{\renewcommand{\literalblockname}{Listing}}

\addto\captionsenglish{\renewcommand{\literalblockcontinuedname}{continued from previous page}}
\addto\captionsenglish{\renewcommand{\literalblockcontinuesname}{continues on next page}}

\addto\extrasenglish{\def\pageautorefname{page}}

\setcounter{tocdepth}{1}



\title{UploadTimer Documentation}
\date{Aug 04, 2017}
\release{0.1}
\author{Matthew J. Shannon}
\newcommand{\sphinxlogo}{\vbox{}}
\renewcommand{\releasename}{Release}
\makeindex

\begin{document}

\maketitle
\sphinxtableofcontents
\phantomsection\label{\detokenize{index::doc}}



\chapter{src package}
\label{\detokenize{src::doc}}\label{\detokenize{src:src-package}}\label{\detokenize{src:welcome-to-uploadtimer-s-documentation}}

\section{Submodules}
\label{\detokenize{src:submodules}}

\section{src.timer module}
\label{\detokenize{src:module-src.timer}}\label{\detokenize{src:src-timer-module}}\index{src.timer (module)}
Track and predict how long an Amazon Cloud upload is going to take.
\index{TimeIt (class in src.timer)}

\begin{fulllineitems}
\phantomsection\label{\detokenize{src:src.timer.TimeIt}}\pysiglinewithargsret{\sphinxbfcode{class }\sphinxcode{src.timer.}\sphinxbfcode{TimeIt}}{\emph{nfiles=100}}{}
Bases: \sphinxcode{object}

Create a time-tracking object.
\index{upload\_data (src.timer.TimeIt attribute)}

\begin{fulllineitems}
\phantomsection\label{\detokenize{src:src.timer.TimeIt.upload_data}}\pysigline{\sphinxbfcode{upload\_data}}
\sphinxstyleemphasis{list} \textendash{} Measurements of file number uploads.

\end{fulllineitems}

\index{time\_data (src.timer.TimeIt attribute)}

\begin{fulllineitems}
\phantomsection\label{\detokenize{src:src.timer.TimeIt.time_data}}\pysigline{\sphinxbfcode{time\_data}}
\sphinxstyleemphasis{list} \textendash{} Measurements of the current system time when
each upload measurement was taken.

\end{fulllineitems}

\index{nfiles (src.timer.TimeIt attribute)}

\begin{fulllineitems}
\phantomsection\label{\detokenize{src:src.timer.TimeIt.nfiles}}\pysigline{\sphinxbfcode{nfiles}}
\sphinxstyleemphasis{int} \textendash{} The total number of files to be uploaded.

\end{fulllineitems}

\index{add\_to\_dataset() (src.timer.TimeIt method)}

\begin{fulllineitems}
\phantomsection\label{\detokenize{src:src.timer.TimeIt.add_to_dataset}}\pysiglinewithargsret{\sphinxbfcode{add\_to\_dataset}}{\emph{file\_number=None}}{}
Record a measurement of current file being uploaded and a timestamp.
\begin{quote}\begin{description}
\item[{Returns}] \leavevmode
True if successful, False otherwise.

\end{description}\end{quote}

\end{fulllineitems}

\index{plot\_it() (src.timer.TimeIt method)}

\begin{fulllineitems}
\phantomsection\label{\detokenize{src:src.timer.TimeIt.plot_it}}\pysiglinewithargsret{\sphinxbfcode{plot\_it}}{\emph{poly=2}, \emph{**kwargs}}{}
Plot data and estimated upload completion time.
\begin{quote}\begin{description}
\item[{Parameters}] \leavevmode\begin{itemize}
\item {} 
\sphinxstyleliteralstrong{poly} (\sphinxstyleliteralemphasis{int}) \textendash{} Degree of polynomial for fit.

\item {} 
\sphinxstyleliteralstrong{**kwargs} \textendash{} Keyword arguments for matplotlib.pyplot.axes.plot.

\end{itemize}

\item[{Returns}] \leavevmode
True if successful, False otherwise.

\end{description}\end{quote}

\end{fulllineitems}

\index{read\_time\_data() (src.timer.TimeIt method)}

\begin{fulllineitems}
\phantomsection\label{\detokenize{src:src.timer.TimeIt.read_time_data}}\pysiglinewithargsret{\sphinxbfcode{read\_time\_data}}{\emph{init=0}, \emph{file\_path='data/timestamps.txt'}}{}
Return a list that holds the timestamps.
\begin{quote}\begin{description}
\item[{Parameters}] \leavevmode\begin{itemize}
\item {} 
\sphinxstyleliteralstrong{init} (\sphinxstyleliteralemphasis{int}) \textendash{} Flag for whether to start from scratch and ignore any
saved previous measurements (set True if desired).

\item {} 
\sphinxstyleliteralstrong{file\_path} (\sphinxstyleliteralemphasis{str}) \textendash{} Path for the timestamps file.

\end{itemize}

\item[{Returns}] \leavevmode
List for holding upload data measurements.

\item[{Return type}] \leavevmode
{\hyperref[\detokenize{src:src.timer.TimeIt.time_data}]{\sphinxcrossref{time\_data}}} (list)

\end{description}\end{quote}

\end{fulllineitems}

\index{read\_upload\_data() (src.timer.TimeIt method)}

\begin{fulllineitems}
\phantomsection\label{\detokenize{src:src.timer.TimeIt.read_upload_data}}\pysiglinewithargsret{\sphinxbfcode{read\_upload\_data}}{\emph{init=0}, \emph{file\_path='data/uploads.txt'}}{}
Return a list that holds the number of files being uploaded.
\begin{quote}\begin{description}
\item[{Parameters}] \leavevmode\begin{itemize}
\item {} 
\sphinxstyleliteralstrong{init} (\sphinxstyleliteralemphasis{int}) \textendash{} Flag for whether to start from scratch and ignore any
saved previous measurements (set True if desired).

\item {} 
\sphinxstyleliteralstrong{file\_path} (\sphinxstyleliteralemphasis{str}) \textendash{} Path for the uploads file.

\end{itemize}

\item[{Returns}] \leavevmode
List for holding upload data measurements.

\item[{Return type}] \leavevmode
{\hyperref[\detokenize{src:src.timer.TimeIt.upload_data}]{\sphinxcrossref{upload\_data}}} (list)

\end{description}\end{quote}

\end{fulllineitems}

\index{time\_data\_to\_dec() (src.timer.TimeIt method)}

\begin{fulllineitems}
\phantomsection\label{\detokenize{src:src.timer.TimeIt.time_data_to_dec}}\pysiglinewithargsret{\sphinxbfcode{time\_data\_to\_dec}}{\emph{time\_data}}{}
Retreive the timestamps, including a decimal version for calculations.
\begin{quote}\begin{description}
\item[{Returns}] \leavevmode
\begin{description}
\item[{Floats of seconds since start of}] \leavevmode
epoch.

\end{description}


\item[{Return type}] \leavevmode
time\_data\_dec (numpy.ndarray)

\end{description}\end{quote}

\end{fulllineitems}

\index{write\_time\_data() (src.timer.TimeIt method)}

\begin{fulllineitems}
\phantomsection\label{\detokenize{src:src.timer.TimeIt.write_time_data}}\pysiglinewithargsret{\sphinxbfcode{write\_time\_data}}{\emph{file\_path='data/timestamps.txt'}}{}
Write the timestamps to a file.
\begin{quote}\begin{description}
\item[{Parameters}] \leavevmode
\sphinxstyleliteralstrong{file\_path} (\sphinxstyleliteralemphasis{str}) \textendash{} Path for the timestamps file.

\item[{Returns}] \leavevmode
True if successful, False otherwise.

\end{description}\end{quote}

\end{fulllineitems}

\index{write\_upload\_data() (src.timer.TimeIt method)}

\begin{fulllineitems}
\phantomsection\label{\detokenize{src:src.timer.TimeIt.write_upload_data}}\pysiglinewithargsret{\sphinxbfcode{write\_upload\_data}}{\emph{file\_path='data/uploads.txt'}}{}
Write the upload data to a file.
\begin{quote}\begin{description}
\item[{Parameters}] \leavevmode
\sphinxstyleliteralstrong{file\_path} (\sphinxstyleliteralemphasis{str}) \textendash{} Path for the uploads file.

\item[{Returns}] \leavevmode
True if successful, False otherwise.

\end{description}\end{quote}

\end{fulllineitems}


\end{fulllineitems}



\section{Module contents}
\label{\detokenize{src:module-src}}\label{\detokenize{src:module-contents}}\index{src (module)}

\chapter{Indices and tables}
\label{\detokenize{index:indices-and-tables}}\begin{itemize}
\item {} 
\DUrole{xref,std,std-ref}{genindex}

\item {} 
\DUrole{xref,std,std-ref}{modindex}

\item {} 
\DUrole{xref,std,std-ref}{search}

\end{itemize}


\renewcommand{\indexname}{Python Module Index}
\begin{sphinxtheindex}
\def\bigletter#1{{\Large\sffamily#1}\nopagebreak\vspace{1mm}}
\bigletter{s}
\item {\sphinxstyleindexentry{src}}\sphinxstyleindexpageref{src:\detokenize{module-src}}
\item {\sphinxstyleindexentry{src.timer}}\sphinxstyleindexpageref{src:\detokenize{module-src.timer}}
\end{sphinxtheindex}

\renewcommand{\indexname}{Index}
\printindex
\end{document}